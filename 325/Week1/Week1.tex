\documentclass[11pt]{article}

    \usepackage[breakable]{tcolorbox}
    \usepackage{parskip} % Stop auto-indenting (to mimic markdown behaviour)
    
    \usepackage{iftex}
    \ifPDFTeX
    	\usepackage[T1]{fontenc}
    	\usepackage{mathpazo}
    \else
    	\usepackage{fontspec}
    \fi

    % Basic figure setup, for now with no caption control since it's done
    % automatically by Pandoc (which extracts ![](path) syntax from Markdown).
    \usepackage{graphicx}
    % Maintain compatibility with old templates. Remove in nbconvert 6.0
    \let\Oldincludegraphics\includegraphics
    % Ensure that by default, figures have no caption (until we provide a
    % proper Figure object with a Caption API and a way to capture that
    % in the conversion process - todo).
    \usepackage{caption}
    \DeclareCaptionFormat{nocaption}{}
    \captionsetup{format=nocaption,aboveskip=0pt,belowskip=0pt}

    \usepackage[Export]{adjustbox} % Used to constrain images to a maximum size
    \adjustboxset{max size={0.9\linewidth}{0.9\paperheight}}
    \usepackage{float}
    \floatplacement{figure}{H} % forces figures to be placed at the correct location
    \usepackage{xcolor} % Allow colors to be defined
    \usepackage{enumerate} % Needed for markdown enumerations to work
    \usepackage{geometry} % Used to adjust the document margins
    \usepackage{amsmath} % Equations
    \usepackage{amssymb} % Equations
    \usepackage{textcomp} % defines textquotesingle
    % Hack from http://tex.stackexchange.com/a/47451/13684:
    \AtBeginDocument{%
        \def\PYZsq{\textquotesingle}% Upright quotes in Pygmentized code
    }
    \usepackage{upquote} % Upright quotes for verbatim code
    \usepackage{eurosym} % defines \euro
    \usepackage[mathletters]{ucs} % Extended unicode (utf-8) support
    \usepackage{fancyvrb} % verbatim replacement that allows latex
    \usepackage{grffile} % extends the file name processing of package graphics 
                         % to support a larger range
    \makeatletter % fix for grffile with XeLaTeX
    \def\Gread@@xetex#1{%
      \IfFileExists{"\Gin@base".bb}%
      {\Gread@eps{\Gin@base.bb}}%
      {\Gread@@xetex@aux#1}%
    }
    \makeatother

    % The hyperref package gives us a pdf with properly built
    % internal navigation ('pdf bookmarks' for the table of contents,
    % internal cross-reference links, web links for URLs, etc.)
    \usepackage{hyperref}
    % The default LaTeX title has an obnoxious amount of whitespace. By default,
    % titling removes some of it. It also provides customization options.
    \usepackage{titling}
    \usepackage{longtable} % longtable support required by pandoc >1.10
    \usepackage{booktabs}  % table support for pandoc > 1.12.2
    \usepackage[inline]{enumitem} % IRkernel/repr support (it uses the enumerate* environment)
    \usepackage[normalem]{ulem} % ulem is needed to support strikethroughs (\sout)
                                % normalem makes italics be italics, not underlines
    \usepackage{mathrsfs}
    

    
    % Colors for the hyperref package
    \definecolor{urlcolor}{rgb}{0,.145,.698}
    \definecolor{linkcolor}{rgb}{.71,0.21,0.01}
    \definecolor{citecolor}{rgb}{.12,.54,.11}

    % ANSI colors
    \definecolor{ansi-black}{HTML}{3E424D}
    \definecolor{ansi-black-intense}{HTML}{282C36}
    \definecolor{ansi-red}{HTML}{E75C58}
    \definecolor{ansi-red-intense}{HTML}{B22B31}
    \definecolor{ansi-green}{HTML}{00A250}
    \definecolor{ansi-green-intense}{HTML}{007427}
    \definecolor{ansi-yellow}{HTML}{DDB62B}
    \definecolor{ansi-yellow-intense}{HTML}{B27D12}
    \definecolor{ansi-blue}{HTML}{208FFB}
    \definecolor{ansi-blue-intense}{HTML}{0065CA}
    \definecolor{ansi-magenta}{HTML}{D160C4}
    \definecolor{ansi-magenta-intense}{HTML}{A03196}
    \definecolor{ansi-cyan}{HTML}{60C6C8}
    \definecolor{ansi-cyan-intense}{HTML}{258F8F}
    \definecolor{ansi-white}{HTML}{C5C1B4}
    \definecolor{ansi-white-intense}{HTML}{A1A6B2}
    \definecolor{ansi-default-inverse-fg}{HTML}{FFFFFF}
    \definecolor{ansi-default-inverse-bg}{HTML}{000000}

    % commands and environments needed by pandoc snippets
    % extracted from the output of `pandoc -s`
    \providecommand{\tightlist}{%
      \setlength{\itemsep}{0pt}\setlength{\parskip}{0pt}}
    \DefineVerbatimEnvironment{Highlighting}{Verbatim}{commandchars=\\\{\}}
    % Add ',fontsize=\small' for more characters per line
    \newenvironment{Shaded}{}{}
    \newcommand{\KeywordTok}[1]{\textcolor[rgb]{0.00,0.44,0.13}{\textbf{{#1}}}}
    \newcommand{\DataTypeTok}[1]{\textcolor[rgb]{0.56,0.13,0.00}{{#1}}}
    \newcommand{\DecValTok}[1]{\textcolor[rgb]{0.25,0.63,0.44}{{#1}}}
    \newcommand{\BaseNTok}[1]{\textcolor[rgb]{0.25,0.63,0.44}{{#1}}}
    \newcommand{\FloatTok}[1]{\textcolor[rgb]{0.25,0.63,0.44}{{#1}}}
    \newcommand{\CharTok}[1]{\textcolor[rgb]{0.25,0.44,0.63}{{#1}}}
    \newcommand{\StringTok}[1]{\textcolor[rgb]{0.25,0.44,0.63}{{#1}}}
    \newcommand{\CommentTok}[1]{\textcolor[rgb]{0.38,0.63,0.69}{\textit{{#1}}}}
    \newcommand{\OtherTok}[1]{\textcolor[rgb]{0.00,0.44,0.13}{{#1}}}
    \newcommand{\AlertTok}[1]{\textcolor[rgb]{1.00,0.00,0.00}{\textbf{{#1}}}}
    \newcommand{\FunctionTok}[1]{\textcolor[rgb]{0.02,0.16,0.49}{{#1}}}
    \newcommand{\RegionMarkerTok}[1]{{#1}}
    \newcommand{\ErrorTok}[1]{\textcolor[rgb]{1.00,0.00,0.00}{\textbf{{#1}}}}
    \newcommand{\NormalTok}[1]{{#1}}
    
    % Additional commands for more recent versions of Pandoc
    \newcommand{\ConstantTok}[1]{\textcolor[rgb]{0.53,0.00,0.00}{{#1}}}
    \newcommand{\SpecialCharTok}[1]{\textcolor[rgb]{0.25,0.44,0.63}{{#1}}}
    \newcommand{\VerbatimStringTok}[1]{\textcolor[rgb]{0.25,0.44,0.63}{{#1}}}
    \newcommand{\SpecialStringTok}[1]{\textcolor[rgb]{0.73,0.40,0.53}{{#1}}}
    \newcommand{\ImportTok}[1]{{#1}}
    \newcommand{\DocumentationTok}[1]{\textcolor[rgb]{0.73,0.13,0.13}{\textit{{#1}}}}
    \newcommand{\AnnotationTok}[1]{\textcolor[rgb]{0.38,0.63,0.69}{\textbf{\textit{{#1}}}}}
    \newcommand{\CommentVarTok}[1]{\textcolor[rgb]{0.38,0.63,0.69}{\textbf{\textit{{#1}}}}}
    \newcommand{\VariableTok}[1]{\textcolor[rgb]{0.10,0.09,0.49}{{#1}}}
    \newcommand{\ControlFlowTok}[1]{\textcolor[rgb]{0.00,0.44,0.13}{\textbf{{#1}}}}
    \newcommand{\OperatorTok}[1]{\textcolor[rgb]{0.40,0.40,0.40}{{#1}}}
    \newcommand{\BuiltInTok}[1]{{#1}}
    \newcommand{\ExtensionTok}[1]{{#1}}
    \newcommand{\PreprocessorTok}[1]{\textcolor[rgb]{0.74,0.48,0.00}{{#1}}}
    \newcommand{\AttributeTok}[1]{\textcolor[rgb]{0.49,0.56,0.16}{{#1}}}
    \newcommand{\InformationTok}[1]{\textcolor[rgb]{0.38,0.63,0.69}{\textbf{\textit{{#1}}}}}
    \newcommand{\WarningTok}[1]{\textcolor[rgb]{0.38,0.63,0.69}{\textbf{\textit{{#1}}}}}
    
    
    % Define a nice break command that doesn't care if a line doesn't already
    % exist.
    \def\br{\hspace*{\fill} \\* }
    % Math Jax compatibility definitions
    \def\gt{>}
    \def\lt{<}
    \let\Oldtex\TeX
    \let\Oldlatex\LaTeX
    \renewcommand{\TeX}{\textrm{\Oldtex}}
    \renewcommand{\LaTeX}{\textrm{\Oldlatex}}
    % Document parameters
    % Document title
    \title{Week1 discussion notes}
    \author{Bora}
    
    
    
    
    
% Pygments definitions
\makeatletter
\def\PY@reset{\let\PY@it=\relax \let\PY@bf=\relax%
    \let\PY@ul=\relax \let\PY@tc=\relax%
    \let\PY@bc=\relax \let\PY@ff=\relax}
\def\PY@tok#1{\csname PY@tok@#1\endcsname}
\def\PY@toks#1+{\ifx\relax#1\empty\else%
    \PY@tok{#1}\expandafter\PY@toks\fi}
\def\PY@do#1{\PY@bc{\PY@tc{\PY@ul{%
    \PY@it{\PY@bf{\PY@ff{#1}}}}}}}
\def\PY#1#2{\PY@reset\PY@toks#1+\relax+\PY@do{#2}}

\expandafter\def\csname PY@tok@w\endcsname{\def\PY@tc##1{\textcolor[rgb]{0.73,0.73,0.73}{##1}}}
\expandafter\def\csname PY@tok@c\endcsname{\let\PY@it=\textit\def\PY@tc##1{\textcolor[rgb]{0.25,0.50,0.50}{##1}}}
\expandafter\def\csname PY@tok@cp\endcsname{\def\PY@tc##1{\textcolor[rgb]{0.74,0.48,0.00}{##1}}}
\expandafter\def\csname PY@tok@k\endcsname{\let\PY@bf=\textbf\def\PY@tc##1{\textcolor[rgb]{0.00,0.50,0.00}{##1}}}
\expandafter\def\csname PY@tok@kp\endcsname{\def\PY@tc##1{\textcolor[rgb]{0.00,0.50,0.00}{##1}}}
\expandafter\def\csname PY@tok@kt\endcsname{\def\PY@tc##1{\textcolor[rgb]{0.69,0.00,0.25}{##1}}}
\expandafter\def\csname PY@tok@o\endcsname{\def\PY@tc##1{\textcolor[rgb]{0.40,0.40,0.40}{##1}}}
\expandafter\def\csname PY@tok@ow\endcsname{\let\PY@bf=\textbf\def\PY@tc##1{\textcolor[rgb]{0.67,0.13,1.00}{##1}}}
\expandafter\def\csname PY@tok@nb\endcsname{\def\PY@tc##1{\textcolor[rgb]{0.00,0.50,0.00}{##1}}}
\expandafter\def\csname PY@tok@nf\endcsname{\def\PY@tc##1{\textcolor[rgb]{0.00,0.00,1.00}{##1}}}
\expandafter\def\csname PY@tok@nc\endcsname{\let\PY@bf=\textbf\def\PY@tc##1{\textcolor[rgb]{0.00,0.00,1.00}{##1}}}
\expandafter\def\csname PY@tok@nn\endcsname{\let\PY@bf=\textbf\def\PY@tc##1{\textcolor[rgb]{0.00,0.00,1.00}{##1}}}
\expandafter\def\csname PY@tok@ne\endcsname{\let\PY@bf=\textbf\def\PY@tc##1{\textcolor[rgb]{0.82,0.25,0.23}{##1}}}
\expandafter\def\csname PY@tok@nv\endcsname{\def\PY@tc##1{\textcolor[rgb]{0.10,0.09,0.49}{##1}}}
\expandafter\def\csname PY@tok@no\endcsname{\def\PY@tc##1{\textcolor[rgb]{0.53,0.00,0.00}{##1}}}
\expandafter\def\csname PY@tok@nl\endcsname{\def\PY@tc##1{\textcolor[rgb]{0.63,0.63,0.00}{##1}}}
\expandafter\def\csname PY@tok@ni\endcsname{\let\PY@bf=\textbf\def\PY@tc##1{\textcolor[rgb]{0.60,0.60,0.60}{##1}}}
\expandafter\def\csname PY@tok@na\endcsname{\def\PY@tc##1{\textcolor[rgb]{0.49,0.56,0.16}{##1}}}
\expandafter\def\csname PY@tok@nt\endcsname{\let\PY@bf=\textbf\def\PY@tc##1{\textcolor[rgb]{0.00,0.50,0.00}{##1}}}
\expandafter\def\csname PY@tok@nd\endcsname{\def\PY@tc##1{\textcolor[rgb]{0.67,0.13,1.00}{##1}}}
\expandafter\def\csname PY@tok@s\endcsname{\def\PY@tc##1{\textcolor[rgb]{0.73,0.13,0.13}{##1}}}
\expandafter\def\csname PY@tok@sd\endcsname{\let\PY@it=\textit\def\PY@tc##1{\textcolor[rgb]{0.73,0.13,0.13}{##1}}}
\expandafter\def\csname PY@tok@si\endcsname{\let\PY@bf=\textbf\def\PY@tc##1{\textcolor[rgb]{0.73,0.40,0.53}{##1}}}
\expandafter\def\csname PY@tok@se\endcsname{\let\PY@bf=\textbf\def\PY@tc##1{\textcolor[rgb]{0.73,0.40,0.13}{##1}}}
\expandafter\def\csname PY@tok@sr\endcsname{\def\PY@tc##1{\textcolor[rgb]{0.73,0.40,0.53}{##1}}}
\expandafter\def\csname PY@tok@ss\endcsname{\def\PY@tc##1{\textcolor[rgb]{0.10,0.09,0.49}{##1}}}
\expandafter\def\csname PY@tok@sx\endcsname{\def\PY@tc##1{\textcolor[rgb]{0.00,0.50,0.00}{##1}}}
\expandafter\def\csname PY@tok@m\endcsname{\def\PY@tc##1{\textcolor[rgb]{0.40,0.40,0.40}{##1}}}
\expandafter\def\csname PY@tok@gh\endcsname{\let\PY@bf=\textbf\def\PY@tc##1{\textcolor[rgb]{0.00,0.00,0.50}{##1}}}
\expandafter\def\csname PY@tok@gu\endcsname{\let\PY@bf=\textbf\def\PY@tc##1{\textcolor[rgb]{0.50,0.00,0.50}{##1}}}
\expandafter\def\csname PY@tok@gd\endcsname{\def\PY@tc##1{\textcolor[rgb]{0.63,0.00,0.00}{##1}}}
\expandafter\def\csname PY@tok@gi\endcsname{\def\PY@tc##1{\textcolor[rgb]{0.00,0.63,0.00}{##1}}}
\expandafter\def\csname PY@tok@gr\endcsname{\def\PY@tc##1{\textcolor[rgb]{1.00,0.00,0.00}{##1}}}
\expandafter\def\csname PY@tok@ge\endcsname{\let\PY@it=\textit}
\expandafter\def\csname PY@tok@gs\endcsname{\let\PY@bf=\textbf}
\expandafter\def\csname PY@tok@gp\endcsname{\let\PY@bf=\textbf\def\PY@tc##1{\textcolor[rgb]{0.00,0.00,0.50}{##1}}}
\expandafter\def\csname PY@tok@go\endcsname{\def\PY@tc##1{\textcolor[rgb]{0.53,0.53,0.53}{##1}}}
\expandafter\def\csname PY@tok@gt\endcsname{\def\PY@tc##1{\textcolor[rgb]{0.00,0.27,0.87}{##1}}}
\expandafter\def\csname PY@tok@err\endcsname{\def\PY@bc##1{\setlength{\fboxsep}{0pt}\fcolorbox[rgb]{1.00,0.00,0.00}{1,1,1}{\strut ##1}}}
\expandafter\def\csname PY@tok@kc\endcsname{\let\PY@bf=\textbf\def\PY@tc##1{\textcolor[rgb]{0.00,0.50,0.00}{##1}}}
\expandafter\def\csname PY@tok@kd\endcsname{\let\PY@bf=\textbf\def\PY@tc##1{\textcolor[rgb]{0.00,0.50,0.00}{##1}}}
\expandafter\def\csname PY@tok@kn\endcsname{\let\PY@bf=\textbf\def\PY@tc##1{\textcolor[rgb]{0.00,0.50,0.00}{##1}}}
\expandafter\def\csname PY@tok@kr\endcsname{\let\PY@bf=\textbf\def\PY@tc##1{\textcolor[rgb]{0.00,0.50,0.00}{##1}}}
\expandafter\def\csname PY@tok@bp\endcsname{\def\PY@tc##1{\textcolor[rgb]{0.00,0.50,0.00}{##1}}}
\expandafter\def\csname PY@tok@fm\endcsname{\def\PY@tc##1{\textcolor[rgb]{0.00,0.00,1.00}{##1}}}
\expandafter\def\csname PY@tok@vc\endcsname{\def\PY@tc##1{\textcolor[rgb]{0.10,0.09,0.49}{##1}}}
\expandafter\def\csname PY@tok@vg\endcsname{\def\PY@tc##1{\textcolor[rgb]{0.10,0.09,0.49}{##1}}}
\expandafter\def\csname PY@tok@vi\endcsname{\def\PY@tc##1{\textcolor[rgb]{0.10,0.09,0.49}{##1}}}
\expandafter\def\csname PY@tok@vm\endcsname{\def\PY@tc##1{\textcolor[rgb]{0.10,0.09,0.49}{##1}}}
\expandafter\def\csname PY@tok@sa\endcsname{\def\PY@tc##1{\textcolor[rgb]{0.73,0.13,0.13}{##1}}}
\expandafter\def\csname PY@tok@sb\endcsname{\def\PY@tc##1{\textcolor[rgb]{0.73,0.13,0.13}{##1}}}
\expandafter\def\csname PY@tok@sc\endcsname{\def\PY@tc##1{\textcolor[rgb]{0.73,0.13,0.13}{##1}}}
\expandafter\def\csname PY@tok@dl\endcsname{\def\PY@tc##1{\textcolor[rgb]{0.73,0.13,0.13}{##1}}}
\expandafter\def\csname PY@tok@s2\endcsname{\def\PY@tc##1{\textcolor[rgb]{0.73,0.13,0.13}{##1}}}
\expandafter\def\csname PY@tok@sh\endcsname{\def\PY@tc##1{\textcolor[rgb]{0.73,0.13,0.13}{##1}}}
\expandafter\def\csname PY@tok@s1\endcsname{\def\PY@tc##1{\textcolor[rgb]{0.73,0.13,0.13}{##1}}}
\expandafter\def\csname PY@tok@mb\endcsname{\def\PY@tc##1{\textcolor[rgb]{0.40,0.40,0.40}{##1}}}
\expandafter\def\csname PY@tok@mf\endcsname{\def\PY@tc##1{\textcolor[rgb]{0.40,0.40,0.40}{##1}}}
\expandafter\def\csname PY@tok@mh\endcsname{\def\PY@tc##1{\textcolor[rgb]{0.40,0.40,0.40}{##1}}}
\expandafter\def\csname PY@tok@mi\endcsname{\def\PY@tc##1{\textcolor[rgb]{0.40,0.40,0.40}{##1}}}
\expandafter\def\csname PY@tok@il\endcsname{\def\PY@tc##1{\textcolor[rgb]{0.40,0.40,0.40}{##1}}}
\expandafter\def\csname PY@tok@mo\endcsname{\def\PY@tc##1{\textcolor[rgb]{0.40,0.40,0.40}{##1}}}
\expandafter\def\csname PY@tok@ch\endcsname{\let\PY@it=\textit\def\PY@tc##1{\textcolor[rgb]{0.25,0.50,0.50}{##1}}}
\expandafter\def\csname PY@tok@cm\endcsname{\let\PY@it=\textit\def\PY@tc##1{\textcolor[rgb]{0.25,0.50,0.50}{##1}}}
\expandafter\def\csname PY@tok@cpf\endcsname{\let\PY@it=\textit\def\PY@tc##1{\textcolor[rgb]{0.25,0.50,0.50}{##1}}}
\expandafter\def\csname PY@tok@c1\endcsname{\let\PY@it=\textit\def\PY@tc##1{\textcolor[rgb]{0.25,0.50,0.50}{##1}}}
\expandafter\def\csname PY@tok@cs\endcsname{\let\PY@it=\textit\def\PY@tc##1{\textcolor[rgb]{0.25,0.50,0.50}{##1}}}

\def\PYZbs{\char`\\}
\def\PYZus{\char`\_}
\def\PYZob{\char`\{}
\def\PYZcb{\char`\}}
\def\PYZca{\char`\^}
\def\PYZam{\char`\&}
\def\PYZlt{\char`\<}
\def\PYZgt{\char`\>}
\def\PYZsh{\char`\#}
\def\PYZpc{\char`\%}
\def\PYZdl{\char`\$}
\def\PYZhy{\char`\-}
\def\PYZsq{\char`\'}
\def\PYZdq{\char`\"}
\def\PYZti{\char`\~}
% for compatibility with earlier versions
\def\PYZat{@}
\def\PYZlb{[}
\def\PYZrb{]}
\makeatother


    % For linebreaks inside Verbatim environment from package fancyvrb. 
    \makeatletter
        \newbox\Wrappedcontinuationbox 
        \newbox\Wrappedvisiblespacebox 
        \newcommand*\Wrappedvisiblespace {\textcolor{red}{\textvisiblespace}} 
        \newcommand*\Wrappedcontinuationsymbol {\textcolor{red}{\llap{\tiny$\m@th\hookrightarrow$}}} 
        \newcommand*\Wrappedcontinuationindent {3ex } 
        \newcommand*\Wrappedafterbreak {\kern\Wrappedcontinuationindent\copy\Wrappedcontinuationbox} 
        % Take advantage of the already applied Pygments mark-up to insert 
        % potential linebreaks for TeX processing. 
        %        {, <, #, %, $, ' and ": go to next line. 
        %        _, }, ^, &, >, - and ~: stay at end of broken line. 
        % Use of \textquotesingle for straight quote. 
        \newcommand*\Wrappedbreaksatspecials {% 
            \def\PYGZus{\discretionary{\char`\_}{\Wrappedafterbreak}{\char`\_}}% 
            \def\PYGZob{\discretionary{}{\Wrappedafterbreak\char`\{}{\char`\{}}% 
            \def\PYGZcb{\discretionary{\char`\}}{\Wrappedafterbreak}{\char`\}}}% 
            \def\PYGZca{\discretionary{\char`\^}{\Wrappedafterbreak}{\char`\^}}% 
            \def\PYGZam{\discretionary{\char`\&}{\Wrappedafterbreak}{\char`\&}}% 
            \def\PYGZlt{\discretionary{}{\Wrappedafterbreak\char`\<}{\char`\<}}% 
            \def\PYGZgt{\discretionary{\char`\>}{\Wrappedafterbreak}{\char`\>}}% 
            \def\PYGZsh{\discretionary{}{\Wrappedafterbreak\char`\#}{\char`\#}}% 
            \def\PYGZpc{\discretionary{}{\Wrappedafterbreak\char`\%}{\char`\%}}% 
            \def\PYGZdl{\discretionary{}{\Wrappedafterbreak\char`\$}{\char`\$}}% 
            \def\PYGZhy{\discretionary{\char`\-}{\Wrappedafterbreak}{\char`\-}}% 
            \def\PYGZsq{\discretionary{}{\Wrappedafterbreak\textquotesingle}{\textquotesingle}}% 
            \def\PYGZdq{\discretionary{}{\Wrappedafterbreak\char`\"}{\char`\"}}% 
            \def\PYGZti{\discretionary{\char`\~}{\Wrappedafterbreak}{\char`\~}}% 
        } 
        % Some characters . , ; ? ! / are not pygmentized. 
        % This macro makes them "active" and they will insert potential linebreaks 
        \newcommand*\Wrappedbreaksatpunct {% 
            \lccode`\~`\.\lowercase{\def~}{\discretionary{\hbox{\char`\.}}{\Wrappedafterbreak}{\hbox{\char`\.}}}% 
            \lccode`\~`\,\lowercase{\def~}{\discretionary{\hbox{\char`\,}}{\Wrappedafterbreak}{\hbox{\char`\,}}}% 
            \lccode`\~`\;\lowercase{\def~}{\discretionary{\hbox{\char`\;}}{\Wrappedafterbreak}{\hbox{\char`\;}}}% 
            \lccode`\~`\:\lowercase{\def~}{\discretionary{\hbox{\char`\:}}{\Wrappedafterbreak}{\hbox{\char`\:}}}% 
            \lccode`\~`\?\lowercase{\def~}{\discretionary{\hbox{\char`\?}}{\Wrappedafterbreak}{\hbox{\char`\?}}}% 
            \lccode`\~`\!\lowercase{\def~}{\discretionary{\hbox{\char`\!}}{\Wrappedafterbreak}{\hbox{\char`\!}}}% 
            \lccode`\~`\/\lowercase{\def~}{\discretionary{\hbox{\char`\/}}{\Wrappedafterbreak}{\hbox{\char`\/}}}% 
            \catcode`\.\active
            \catcode`\,\active 
            \catcode`\;\active
            \catcode`\:\active
            \catcode`\?\active
            \catcode`\!\active
            \catcode`\/\active 
            \lccode`\~`\~ 	
        }
    \makeatother

    \let\OriginalVerbatim=\Verbatim
    \makeatletter
    \renewcommand{\Verbatim}[1][1]{%
        %\parskip\z@skip
        \sbox\Wrappedcontinuationbox {\Wrappedcontinuationsymbol}%
        \sbox\Wrappedvisiblespacebox {\FV@SetupFont\Wrappedvisiblespace}%
        \def\FancyVerbFormatLine ##1{\hsize\linewidth
            \vtop{\raggedright\hyphenpenalty\z@\exhyphenpenalty\z@
                \doublehyphendemerits\z@\finalhyphendemerits\z@
                \strut ##1\strut}%
        }%
        % If the linebreak is at a space, the latter will be displayed as visible
        % space at end of first line, and a continuation symbol starts next line.
        % Stretch/shrink are however usually zero for typewriter font.
        \def\FV@Space {%
            \nobreak\hskip\z@ plus\fontdimen3\font minus\fontdimen4\font
            \discretionary{\copy\Wrappedvisiblespacebox}{\Wrappedafterbreak}
            {\kern\fontdimen2\font}%
        }%
        
        % Allow breaks at special characters using \PYG... macros.
        \Wrappedbreaksatspecials
        % Breaks at punctuation characters . , ; ? ! and / need catcode=\active 	
        \OriginalVerbatim[#1,codes*=\Wrappedbreaksatpunct]%
    }
    \makeatother

    % Exact colors from NB
    \definecolor{incolor}{HTML}{303F9F}
    \definecolor{outcolor}{HTML}{D84315}
    \definecolor{cellborder}{HTML}{CFCFCF}
    \definecolor{cellbackground}{HTML}{F7F7F7}
    
    % prompt
    \makeatletter
    \newcommand{\boxspacing}{\kern\kvtcb@left@rule\kern\kvtcb@boxsep}
    \makeatother
    \newcommand{\prompt}[4]{
        \ttfamily\llap{{\color{#2}[#3]:\hspace{3pt}#4}}\vspace{-\baselineskip}
    }
    

    
    % Prevent overflowing lines due to hard-to-break entities
    \sloppy 
    % Setup hyperref package
    \hypersetup{
      breaklinks=true,  % so long urls are correctly broken across lines
      colorlinks=true,
      urlcolor=urlcolor,
      linkcolor=linkcolor,
      citecolor=citecolor,
      }
    % Slightly bigger margins than the latex defaults
    
    \geometry{verbose,tmargin=1in,bmargin=1in,lmargin=1in,rmargin=1in}
    
    

\begin{document}
    
    \maketitle
    
    

    
    \hypertarget{week-1-math-review}{%
\section{Week 1: Math review}\label{week-1-math-review}}

\hypertarget{some-very-basic-real-analysis}{%
\subsection{Some very basic real
analysis}\label{some-very-basic-real-analysis}}

As you know, functions can be represented by either an infinite or
finite sum of other functions. In the former case, this is true if the
infinite series evaluates to something finite - it is convergent rather
than divergent. In class I only went through the Taylor expansion where
the convergence property is rather obvious. Let's now review the more
complete picture.

\textbf{1.1} I'm assuming you all know and remember how to deal with a
series of numbers. We extend these series such that each element of the
sum is a function, \(u_i\), of variable \(x\). The notion of a partial
sum is important here: \[
\sigma_n(x):=\sum^n_iu_i(x)
\] Then the infinite sum is \emph{defined} as the limit of infinite
\(n\), \(\sigma_{n\to \infty}(x)=f(x)\). 
\begin{tcolorbox}[title=Convergence musings]
If this limit exists, the sum
is \emph{convergent}. But how does the existence of this limit depend on
the variable \(x\)? Well, if it doesn't then the series is said to be
\emph{uniformly convergent}.

\begin{quote}
Let's say this exact same thing formally: If \(\forall \epsilon>0\) and
\(x\in[a,b]\), \(\exists N\in \mathbb{Z}_+\) such that
\(|f(x)-\sigma_{n>N}(x)|<\epsilon\), the series converges to \(f(x)\)
uniformly on interval \([a,b]\). (Read as if for all positive
\(\epsilon\) and \(x\) in interval {[}a,b{]} there exists a positive
integer \(N\)\ldots)
\end{quote}

The concept of non-uniform convergence comes up usually near the
boundaries of the interval of convergence where there must be some
constraint on \(x\) that renders the sum finite.

\begin{quote}
I'll remind you that series can also either be absolutely or
conditionally convergent. A series, \(\sum_i^\infty u_i(x)\) is
absolutely convergent if \(\sum_i^\infty |u_i(x)|<\infty\). If a series
is convergent but not absolutely convergent, it is conditionally
convergent. \emph{Absolute/conditional convergence is a distinct notion
from uniform convergence!}
\end{quote}

As you probably recall, there are tests of all kinds of convergence. Two
popular tests for uniform convergence is the Weierstrass M test (very
simple to prove and implement) and Abel's test (rather subtle but
important for power series).

\textbf{Uniform convergence is very important to check because it
implies \(f(x)\) and \(u_i(x)\) are at least continuous and the series
can be integrated or differentiated term-wise.}
\end{tcolorbox}

\textbf{1.2} The first note is to remind you about the notion of
convergence which you should always have in the back of your mind.
Practically, what's more relevant is constructing these function series
in a useful way. In physics useful will typically mean

\begin{itemize}
\tightlist
\item
  Approximate integrals by replacing the integrand with a series
  expansion
\item
  Solve ODEs by choosing an appropriate power series expansion of the
  solution (this is where you care about convergence).
\item
  Approximate ODEs by perturbation series
\end{itemize}

Suppose a function is represented by a uniformly convergent series \[
f(x)=\sum_i^\infty u_i(x) .
\] on some interval. By Taylor's theorem, we know what the $u_i(x)$ are:
$$
u_i(x) = \frac{f^{(i)}(a)}{i!}(x-a)^i.
$$

\begin{tcolorbox}[title=Taylor expansions]

How do we actually figure out \(u_i(x)\) for a
given \(f(x)\)? Suppose the function is such that I can differentiate it
as much as I want. Consider \[
\int_a^xdx_1f^{(n)}(x_1)=f^{(n-1)}(x)-f^{(n-1)}(a)
\] If I keep integrating, I will eventually reach \(f(x)-f(a)\): \[
\int_a^xdx_n\int_{a}^{x_n}dx_{n-1}...\int_a^{x_2}dx_1f^{(n)}(x_1)=f(x)-f(a)-(x-a)f'(a)-(x-a)^2f''(a)/2!+...
\] Define the LHS to be \(R_n(x)\), the remainder and solve for
\(f(x)\): \[
f(x)=f(a)+(x-a)f'(a)+(x-a)^2f''(a)/2!+R_n(x)
\] 
The remainder can be thought of as the error incurred by truncating
at order \(n\). The condition for the convergence of a Taylor series to
the function is simple, we just need \(R_{n\to\infty}=0\). Finally, we
have as the Taylor expansion: \[
f(x) = \sum^\infty_{i=0}\frac{(x-a)^n}{n!}f^{(n)}(a)
\] If you choose to truncate at some finite order we say the expansion
is accurate to that order, for small \(x\).

\begin{quote}
The form of the remainder is rather ugly and probably unfamiliar. By
applying the
\href{https://en.wikipedia.org/wiki/Mean_value_theorem}{mean value
theorem} \(n\) times we get the more familiar \[
R_n = \frac{(x-a)^n}{n!} f^{(n)}(\zeta)
\] for \(\zeta\in[a, x]\). As an exercise, show that this is true. The
mean value theorem is very useful and you should know how to apply it.
\end{quote}
\end{tcolorbox}




The named functions like the exponential, trig functions, logarithm etc.
are defined by their Taylor expansions around \(a=0\). So, \(\exp(x)\)
is literally \(\sum_{i=0}^\infty x^n/n!\)

\begin{quote}
Taylor series are an example of what are called power series where the
\(x\) dependence of the terms is monomial. Power series can be easily
checked for convergence. For coefficients \(a_n\), if \[
\lim_{n\to \infty}\left|\frac{a_{n+1}}{a_n} \right| = \frac{1}{R}
\] the series converges on at least \((-R, R)\) and possibly
\([-R, R]\). It converges uniformly and absolutely inside
\((-R+\epsilon, R-\epsilon)\). That is, power series may be term wise
integrated and differentiated inside their interval of convergence.
\end{quote}

To test your understanding, prove that power series expansions of
functions are unique. This is an extremely important technical result.

\textbf{1.3} Here are some important techniques you might find useful:

\begin{itemize}
\item
  \emph{Cauchy product:} Consider
  \(f(x)g(x)=\sum_i a_i x^i\times \sum_jb_j x^j\). The product of power
  series is a \emph{convolution}: \[
   f(x)g(x)= \sum_{i}^\infty\sum_j^i a_j b_{i-j}x^i
  \]
\item
  \emph{Binomial expansion:} Often you'll com across powers of sums.
  These are finite special power series where the coefficient are the
  binomial coefficients (n choose r): \[
  (x+a)^n = \sum_k^n C_{n}^k x^k a^{n-k}
  \]
\item
  \emph{Inversion}: In algebraic manipulations it might be necessary to
  invert power series. Typically it's sufficient to brute force things.
  Suppose you're trying to solve \(y = \sum_i^\infty a_i x^i\) for
  \(x\). Because expansions are unique, we know \[
  x=\sum_i b_i y^i\implies x = \sum_{i}b_i \left(\sum_j a_j x^j\right)^i
  \] This is a mess but you can tackle it term by term: \(b_1=1/a_1,\)
  \(b_2=-a_2/a_1^3\) and so on.
\item
  \emph{\href{https://mathworld.wolfram.com/EulersSeriesTransformation.html}{Euler
  transformation}}: This improves convergence of alternating series and
  might make finding the convergence radius easier.
\item
  \emph{\href{https://en.wikipedia.org/wiki/Partial_fraction_decomposition}{Partial
  fraction decomposition}}: This is a very commonly employed technique
  especially when multiplying power series and integrating ratios.
\end{itemize}

\textbf{1.4} It's great to have power series expansions - Uniqueness of
such expansions is certainly very powerful in solving ODEs (formally)
exactly and the fact that we can bound error means we can get the
desired accuracy with sufficient effort. What if these aren't
immediately granted? Are non-unique expansions with unknown error of any
use to us? Can we make use of divergent expansions?

\textbf{It turns out such expansions are even more useful:} These are
the so-called \emph{asymptotic expansions} of which the (truncated)
Taylor expansion is a particular example. Regarding the latter question,
it is often the case that divergent series better approximate certain
functions than convergent ones at the same order (Bessel functions
usually used to illustrate this). The idea is that as long as you know
of a small parameter, you can expand things and make progress regardless
of the analytical details. You will be doing physics this way in no
time!

\begin{quote}
Note the leading terms in \(\sin x\sim x-x^3/6\) near \(x=0\). But let's
also look at \(\sin x\sim x+2 x^2\). Near \(x=0\) it's hard to tell the
two approximations apart, though third order Taylor is better
(obviously). Thus, \(\sin\) doesn't have unique asymptotic expansions
but do you \emph{really} care near \(x=0\)?
\end{quote}

    \begin{tcolorbox}[breakable, size=fbox, boxrule=1pt, pad at break*=1mm,colback=cellbackground, colframe=cellborder]
\prompt{In}{incolor}{1}{\boxspacing}
\begin{Verbatim}[commandchars=\\\{\}]
\PY{k+kn}{import} \PY{n+nn}{matplotlib}\PY{n+nn}{.}\PY{n+nn}{pyplot} \PY{k}{as} \PY{n+nn}{plt}
\PY{k+kn}{import} \PY{n+nn}{numpy} \PY{k}{as} \PY{n+nn}{np}
\PY{n}{x} \PY{o}{=} \PY{n}{np}\PY{o}{.}\PY{n}{linspace}\PY{p}{(}\PY{o}{\PYZhy{}}\PY{o}{.}\PY{l+m+mi}{05}\PY{p}{,} \PY{o}{.}\PY{l+m+mi}{05}\PY{p}{,} \PY{l+m+mi}{200}\PY{p}{)}
\PY{n}{plt}\PY{o}{.}\PY{n}{plot}\PY{p}{(}\PY{n}{x}\PY{p}{,} \PY{n}{np}\PY{o}{.}\PY{n}{sin}\PY{p}{(}\PY{n}{x}\PY{p}{)}\PY{p}{,} \PY{n}{label}\PY{o}{=}\PY{l+s+sa}{r}\PY{l+s+s1}{\PYZsq{}}\PY{l+s+s1}{\PYZdl{}sin x\PYZdl{}}\PY{l+s+s1}{\PYZsq{}}\PY{p}{)}
\PY{n}{plt}\PY{o}{.}\PY{n}{plot}\PY{p}{(}\PY{n}{x}\PY{p}{,} \PY{n}{x}\PY{o}{\PYZhy{}}\PY{n}{x}\PY{o}{*}\PY{o}{*}\PY{l+m+mi}{3}\PY{o}{/}\PY{l+m+mi}{6}\PY{p}{,} \PY{n}{label}\PY{o}{=}\PY{l+s+sa}{r}\PY{l+s+s1}{\PYZsq{}}\PY{l+s+s1}{Third order Taylor}\PY{l+s+s1}{\PYZsq{}}\PY{p}{)}
\PY{n}{plt}\PY{o}{.}\PY{n}{plot}\PY{p}{(}\PY{n}{x}\PY{p}{,} \PY{n}{x}\PY{o}{+}\PY{l+m+mi}{2}\PY{o}{*}\PY{n}{x}\PY{o}{*}\PY{o}{*}\PY{l+m+mi}{2}\PY{p}{,} \PY{n}{label}\PY{o}{=}\PY{l+s+sa}{r}\PY{l+s+s1}{\PYZsq{}}\PY{l+s+s1}{Second order asymptotic}\PY{l+s+s1}{\PYZsq{}}\PY{p}{)}
\PY{n}{plt}\PY{o}{.}\PY{n}{legend}\PY{p}{(}\PY{p}{)}
\PY{n}{plt}\PY{o}{.}\PY{n}{grid}\PY{p}{(}\PY{p}{)}
\end{Verbatim}
\end{tcolorbox}

    \begin{center}
    \adjustimage{max size={0.9\linewidth}{0.9\paperheight}}{Week1_1_0.png}
    \end{center}
    { \hspace*{\fill} \\}
    
    \hypertarget{differential-vector-calculus}{%
\subsection{Differential vector
calculus}\label{differential-vector-calculus}}

\textbf{2.1} Consider a set together with two operations - addition and
scalar multiplication. Roughly, such a set is a vector space with each
element called a vector if the sum of of two vectors is again a vector
and multiplication by a scalar rescales the vector. There is an additive
inverse (\(-\vec v=\vec v^{-1}\)), additive identity (\(\vec 0\)) and
scalar identity (\(1\)). There are two more operators we may consider:

\begin{itemize}
\tightlist
\item
  The scalar product which assigns a number to two vectors (technically
  one vector and one co-vector).
\item
  The cross product takes two vectors into a vector normal to the plane
  spanned by the two vectors where normality is defined relative to the
  inner product.
\end{itemize}

Let's focus on vector in Euclidean space, \(\mathbb{R}^d\) where a
vector is, morally speaking, a direction and a magnitude. In this simple
review we choose the intuitive basis \[
\vec v =\sum_{i=1}^dv_i\hat e_i
\] with \(\hat e_i\) the unit vector pointing in direction \(i\) and
\(v_i\) the component of the vector in that direction. The basis vectors
are such that \(\hat e_i\cdot \hat e_j = \delta_{ij}\). Two such vectors
for which this is true are said to be orthonormal.

\textbf{2.2} We live in \(\mathbb R^3\) so a vector at \emph{a point}
(technically we say it is over the point) in this space is
\(\vec v = a \hat x+ b\hat y+c \hat z\) (in rectangular coordinates). If
we assign such a vector to each point in \(\mathbb R^3\) we get a vector
field on \(\mathbb R^3\), \[
\vec F(x,y,z) = a(x,y,z)\hat x+b(x,y,z)\hat y+c(x,y,z)\hat z.
\] That is, it is a vector valued function
\(F:\mathbb R^3\to \mathbb R^3\). This function is assumed to be smooth.

\begin{tcolorbox}[title=Some fancy geometry words]
\emph{A vector field is a
section of the tangent bundle over some manifold.} This is the
language of differential geometry - you might find MATH 481 to be useful
if you want to get a better appreciation for the vector calculus you'll
be doing. 
\end{tcolorbox}

\textbf{2.3} At the end of the day, we want to study how things change
so we should think about how to differentiate things. Note first that a
vector field may already be the derivative of a scalar function. A
derivation that maps a scalar to a vector at each point is a gradient,
\(\vec \nabla=\partial_x \hat x+\partial_y\hat y+\partial_z\hat z\).

If \(\vec F(x,y,z) =\vec \nabla U(x,y,z)\) we say the vector field is a
conserving field. Tyically, the vector fields we consider will be
gradients of scalar functions. You will see why in a week or two.

The gradient of a scalar function vanishes at what are called critical
points. Such points are either local maxima, local minima or saddles.
The classification of critical points is more nuanced in higher
dimensions (\(d>2\)) than the second derivative test (see 2.4).

With $\vec \nabla $ being a vector, we can consider inner and cross
products with vector fields: \[
\text{curl}(F) = \vec\nabla\times \vec F(x,y,z)
\]
$$
\text{div}(F) = \vec\nabla \cdot F(x,y,z)
$$

The curl tells you how quickly a vector field twists while the
divergence tells you how something about the flow rate (flux) through
some imaginary surface. This is all stuff you should know\ldots{}

\textbf{2.4} We can also consider various second derivatives. The most
important is the Laplacian of a scalar function: \[
\nabla\cdot(\nabla U(x,y,z)):=\nabla^2U(x,y,z).
\] \textgreater You'll probably see me write \(\Delta\) for the
Laplacian.

\begin{quote}
The geometric analog of the cross second derivative
\(\partial^2/\partial x\partial y\) is the Hessian. It's defined as the
Jacobian of a gradient. In components, \[
H_{ij}f=\frac{\partial^2 f}{\partial x^i\partial x^j}.
\] We won't use the Hessian though it has an important role in dynamics,
in particular it can classify critical points (there is an entire field
of geometry built around this object called Morse theory). It is morally
the `second derivative' test - you look at the sign of the Hessian
eigenvalues (which you'll learn about).
\end{quote}

\hypertarget{reading-recommendations}{%
\subsection{Reading recommendations}\label{reading-recommendations}}

\begin{itemize}
\item
  \emph{Mathematical methods for physicists} by Arfken and Weber is a
  great resource for learning and brushing up on the applied math used
  in physics. I've used it as a guide in writing the stuff on series.

  \begin{itemize}
  \tightlist
  \item
    If you \emph{really} want to learn `calculus', \emph{Introduction to
    Real Analysis} by Bartle and Sherbert is pretty standard.
  \end{itemize}
\item
  \emph{\href{https://link.springer.com/book/10.1007/978-0-387-87765-5}{Introduction
  to the foundation of applied math}} by Mark Holmes is great for
  learning about how to \emph{actually} solve ODEs. There's a bit of CM
  in it too. I would read the first few chapters ASAP.
\end{itemize}

\begin{quote}
You can get all springer books for free by using the UIUC library proxy
or a VPN. Just go on springer link.
\end{quote}




    % Add a bibliography block to the postdoc
    
    
    
\end{document}
