\documentclass{article}
\usepackage{amsmath}
\usepackage{amssymb}
\usepackage{graphicx}
\usepackage{amsthm}
\usepackage{empheq}
\usepackage{float}
\usepackage[utf8]{inputenc}
\usepackage{amssymb}
\usepackage[most]{tcolorbox}
\usepackage{hyperref}
\newtheorem*{theorem}{Thm}
\newtheorem*{proposition}{Prop}
\theoremstyle{definition}
\newtheorem*{definition}{Defn}

\theoremstyle{remark}
\newtheorem*{remark}{Remark}

\theoremstyle{remark}
\newtheorem*{example}{Example}


\newtcbox{\mymath}[1][]{%
    nobeforeafter, math upper, tcbox raise base,
    enhanced, colframe=blue!30!black,
    colback=blue!30, boxrule=1pt,
    #1}
\title{Green function basics}




\begin{document}

\maketitle

Let $\mathcal L$ be a second order linear differential operator. We want to solve the inhomogeneous problem
$$
\mathcal L y(t) = f(t).
$$
We can make progress by noting that $\exists G(t, t')$ s.t.
$$
\mathcal L G(t, t') = \delta(t-t')
$$
That such a $G$ exists is a property of homogeneous second order linear differential equations and is independent of $f$. 

\begin{tcolorbox}
	Physically, $G(t,t')$ is the response of the system to a singular impulse at $t=t'$.

If the total energy is conserved dynamics are invariant under $t\to t+t_0$. In this case we write
$$
G(t,t') = G(t-t');
$$ 
the response function depends only on the interval between time $t$ and time of impulse $t'$.
\end{tcolorbox}
We have the following fact:
$$
\int dt' \mathcal L G(t-t') f(t)  = \mathcal L \left\{\int dt' G(t-t')f(t)\right\} = f(t)
$$
where I have used the linearity of $\mathcal L$ to pull it out of the integral. Clearly, 
$$
y(t) = \int dt' G(t-t')f(t)
$$
is our solution. The solution is the \emph{convolution}
$$
y(t) = G(t-t')\star f(t).
$$
Then, if $G$ is determined, we have a way of obtaining the solution relatively painlessly. 

For simple problems that you will encounter in this class, the Green function can be determined systematically by Fourier transforming $\mathcal L$ (which renders the differential operator algebraic):
$$
 G(\omega) = \frac{1}{\mathcal L_\omega}
$$ 
The solution in $\omega$ space is simply the product
$$
Y(\omega) = G(\omega)F(\omega).
$$
The last step is to look up the inverse Fourier transform of this $Y$.

\begin{tcolorbox}
	In more complicated boundary value problems a systematic approach is to expand $G$ in a function basis suitable for the geometry of the boundary (or boundaries). If $\mathcal L$ can be diagonalized, this eigenbasis is the suitable basis to construct the Green function. 

	You will learn this more general technique (which is the subject of what is called Fredholm theory) in E\&M and later in quantum field theory.
\end{tcolorbox}



\end{document}

\begin{empheq}[box=\tcbhighmath]{align*}
    F_r&=m(\ddot r-r\dot \theta^2-r\dot\phi^2\sin^2\theta)\\
  F_\theta&=m(2\dot r\dot \theta+r\ddot \theta-r\dot\phi^2\sin\theta\cos\theta)\\
  F_\phi&=m(r\ddot \phi \sin \theta +2\dot r \dot \phi \sin \theta +2r\dot\theta \dot \phi\cos\theta)
\end{empheq}

   