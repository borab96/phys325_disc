\documentclass{article}
\usepackage{amsmath}
\usepackage{amssymb}
\usepackage{graphicx}
\usepackage{amsthm}
\usepackage{empheq}
\usepackage{float}
\usepackage[utf8]{inputenc}
\usepackage{amssymb}
\usepackage[most]{tcolorbox}

\newtheorem*{theorem}{Thm}
\newtheorem*{proposition}{Prop}
\theoremstyle{definition}
\newtheorem*{definition}{Defn}

\theoremstyle{remark}
\newtheorem*{remark}{Remark}

\theoremstyle{example}
\newtheorem*{example}{Ex}

\newtcbox{\mymath}[1][]{%
    nobeforeafter, math upper, tcbox raise base,
    enhanced, colframe=blue!30!black,
    colback=blue!30, boxrule=1pt,
    #1}
\title{A very specific review and some coupled oscialltions}

\begin{document}

\maketitle

Things to recall:
\begin{itemize}
  \item $E=T+U$, the total energy, is a constant of motion (in closed mechanical systems). The dynamical data under this constaint is encoded in $\dot E=0$.
  \begin{example}[Simple pendulum] We have the total energy $\frac{1}{2}m l^2\dot \theta^2-mgl\cos\theta$. The constancy of this quantitiy implies
    $$
    \dot E =0 \implies ml^2 \ddot \theta\dot\theta+ mgl \sin \theta \dot \theta = 0
    $$
    which yields the familiar
    $$
    \ddot \theta+\frac{g}{l}\sin\theta =0
    $$
    as the equation of motion. In this case the statement that total energy is conserved contains all the dynamical information of the system.
  \end{example}
  \item A rolling object is said to not slip if its transnational velocity is matched by the tangential velocity of rolling: $\dot q_{center} = r\dot \phi$. 
  \item What we mean when we say \emph{solving} a classical mechanical system (in the Lagrangian formalism): For $N$ degrees of freedom identify $N$ generalized coordinates. Work in coordinate systems that are convenient for the problem. Once the coordinate system is identified, $\{q_i,\dot q_i\}_{i=1,...,N}$, construct the Lagrangian
  $$
  L[q(t),\dot q(t);t] = T-U
  $$
  and solve
  $$
  \frac{\partial L}{\partial q_i} = \frac{d}{dt}\frac{\partial L}{\partial \dot q_i}
  $$
  for each coordinate.
  \item When we say frequency of oscillation, we specifically mean the oscillation frequency of the harmonic oscillator. This is the $\omega$ that appears in
  $$
  L_{SHO} = \frac{1}{2}m\dot q^2-\underbrace{\frac{1}{2}m\omega^2 q^2}_{U_{SHO}}
  $$
  When you are asked to find the frequency of small oscillations, you need to figure out the parameters for which 
  $$
  U_{\text{your system}} = U_{SHO}+\mathcal O(q^3)
  $$
\end{itemize}

\end{document}

\begin{empheq}[box=\tcbhighmath]{align*}
    F_r&=m(\ddot r-r\dot \theta^2-r\dot\phi^2\sin^2\theta)\\
  F_\theta&=m(2\dot r\dot \theta+r\ddot \theta-r\dot\phi^2\sin\theta\cos\theta)\\
  F_\phi&=m(r\ddot \phi \sin \theta +2\dot r \dot \phi \sin \theta +2r\dot\theta \dot \phi\cos\theta)
\end{empheq}
