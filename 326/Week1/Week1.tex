\documentclass{article}
\usepackage{amsmath}
\usepackage{amssymb}
\usepackage{graphicx}
\usepackage{amsthm}
\usepackage{empheq}
\usepackage{float}
\usepackage[utf8]{inputenc}
\usepackage{amssymb}
\usepackage[most]{tcolorbox}

\newtheorem*{theorem}{Thm}
\newtheorem*{proposition}{Prop}
\theoremstyle{definition}
\newtheorem*{definition}{Defn}

\theoremstyle{remark}
\newtheorem*{remark}{Remark}

\theoremstyle{example}
\newtheorem*{example}{Ex}

\newtcbox{\mymath}[1][]{%
    nobeforeafter, math upper, tcbox raise base,
    enhanced, colframe=blue!30!black,
    colback=blue!30, boxrule=1pt,
    #1}
\title{A very specific review and some coupled oscialltions}

\begin{document}

\maketitle

Things to recall:
\begin{itemize}
  \item $E=T+U$, the total energy, is a constant of motion (in closed mechanical systems). The dynamical data under this constraint is encoded in $\dot E=0$.
  \begin{example}[Simple pendulum] We have the total energy $\frac{1}{2}m l^2\dot \theta^2-mgl\cos\theta$. The constancy of this quantity implies
    $$
    \dot E =0 \implies ml^2 \ddot \theta\dot\theta+ mgl \sin \theta \dot \theta = 0
    $$
    which yields the familiar
    $$
    \ddot \theta+\frac{g}{l}\sin\theta =0
    $$
    as the equation of motion. In this case the statement that total energy is conserved contains all the dynamical information of the system.
  \end{example}
  \item A rolling object is said to not slip if its transnational velocity is matched by the tangential velocity of rolling: $\dot q_{center} = r\dot \phi$. 
  \item What we mean when we say \emph{solving} a classical mechanical system (in the Lagrangian formalism): For $N$ degrees of freedom identify $N$ generalized coordinates. Work in coordinate systems that are convenient for the problem. Once the coordinate system is identified, $\{q_i,\dot q_i\}_{i=1,...,N}$, construct the Lagrangian
  $$
  L[q(t),\dot q(t);t] = T-U
  $$
  and solve
  $$
  \frac{\partial L}{\partial q_i} = \frac{d}{dt}\frac{\partial L}{\partial \dot q_i}
  $$
  for each coordinate.
  \item When we say frequency of oscillation, we specifically mean the oscillation frequency of the harmonic oscillator. This is the $\omega$ that appears in
  $$
  L_{SHO} = \frac{1}{2}m\dot q^2-\underbrace{\frac{1}{2}m\omega^2 q^2}_{U_{SHO}}
  $$
  When you are asked to find the frequency of small oscillations, you need to figure out the parameters for which 
  $$
  U_{\text{your system}} = U_{SHO}+\mathcal O(q^3)
  $$
\end{itemize}

\subsection*{Binary oscillator}

Consider $N$ \emph{linearly} coupled bodies with Lagrangian
$$
\frac{1}{2}\sum_i^N \dot x_i^2 m_i - \frac{1}{2}\sum_{i,j}^N k_{i}(x_i-x_j)^2.
$$
We have a notion of "$F=ma$" though in matrix form (tensorial if you want to be more precise):
\begin{equation}\label{eq:EOM}
M_{ij}\partial_t^2x_j = -K_{ij}x_j
\end{equation}
where the mass matrix entries are given by
$$
M_{ij} = \frac{\partial^2 T}{\partial\dot x_i\partial x_j}
$$
and correspondingly the matrix entries of the stiffness 
$$
K_{ij} = \frac{\partial^2 V}{\partial\dot x_i\partial x_j}.
$$
You might imagine $K_{ij}$ to be read as the stiffness of the spring coupling degrees of freedom $i$ and $j$. Keep in mind, however, that this intuition is basis dependent. 

Since our EOM (Eq.~\ref{eq:EOM}) is linear, we can go ahead and Fourier transform:
 $$
 -M_{ij}\omega^2x_j(\omega) = -K_{ij}x_j(\omega)\implies (M_{ij}\omega^2-K_{ij})x_{j}(\omega)=0
 $$
 Here is a theorem you will come across in liner algebra:
\begin{theorem}
The following are equivalent
\begin{enumerate}
  \item $\dim\text{Ker} (M_{ij}\omega^2-K_{ij})>0$
  \item $\det (M_{ij}\omega^2-K_{ij})=0$
\end{enumerate}
\end{theorem}
This tells you that motion will be \emph{non trivial} only if $\det (M_{ij}\omega^2-K_{ij})=0$ so we must find $\omega$s that satisfy this equation. In practice, you will be able to construct these $\mathbf K$ and $\mathbf M$ matrices explicitly.

\subsubsection*{Formal details}

In my more general example, I proceed like so:

First, assume $\mathbf M$ is diagonal, as it almost always is, and write $M_{ij}=m_j\delta_{ij}=m_i$. Let us take $\Lambda$ to the matrix of ordered and orthonormalized eigenvectors of $\mathbf K$. Then, recalling that $\det \mathbf A = \prod_i \text{Eigenvalues}(A)_i$
\begin{align*}
  \prod_{i}(\omega^2m_i -(\mathbf\Lambda\cdot \mathbf K\mathbf\Lambda^T)_{ii})=0
\end{align*}
Thus, operationally, we just need to diagonalize the stiffness matrix to deteriming the $\omega$'s.


\end{document}

\begin{empheq}[box=\tcbhighmath]{align*}
    F_r&=m(\ddot r-r\dot \theta^2-r\dot\phi^2\sin^2\theta)\\
  F_\theta&=m(2\dot r\dot \theta+r\ddot \theta-r\dot\phi^2\sin\theta\cos\theta)\\
  F_\phi&=m(r\ddot \phi \sin \theta +2\dot r \dot \phi \sin \theta +2r\dot\theta \dot \phi\cos\theta)
\end{empheq}
