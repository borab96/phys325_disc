\documentclass{article}
\usepackage{amsmath}
\usepackage{amssymb}
\usepackage{graphicx}
\usepackage{amsthm}
\usepackage{empheq}
\usepackage{float}
\usepackage[utf8]{inputenc}
\usepackage{amssymb}
\usepackage[most]{tcolorbox}
\usepackage{hyperref}
\newtheorem*{theorem}{Thm}
\newtheorem*{proposition}{Prop}
\theoremstyle{definition}
\newtheorem*{definition}{Defn}

\theoremstyle{remark}
\newtheorem*{remark}{Remark}

\theoremstyle{remark}
\newtheorem*{example}{Example}


\newtcbox{\mymath}[1][]{%
    nobeforeafter, math upper, tcbox raise base,
    enhanced, colframe=blue!30!black,
    colback=blue!30, boxrule=1pt,
    #1}
\title{Geodesics}




\begin{document}

\maketitle

\section{A rapid intro to differential geometry}

We often want to be able to do calculus on spaces that are not just $\mathbb{R}^n$. Consider for instance the problem of determining the minimum distance between two points on a sphere. 

It is possible to express spaces on which it is meaningful to do calculus as subsets of Euclidean space of sufficiently high dimension (Whitney embedding theorem). As it turns out, this is a very boring way of doing geometry so we take a more intrinsic perspective:

\begin{definition}
  A manifold of dimension $n$ a (topological) space that is locally modeled on $\mathbb R^n$ (looks like flat space if you zoom in enough). The union of these Euclidean patches is the manifold. If the transitions between the patches are given by smooth functions, the manifold is smooth (they should be at least continuous in which case you have a topological manifold).
\end{definition}

\begin{example}
$S^n$, the n-dimensional sphere is a manifold. Think for instance of $S^2$: It is impossible to flatten the 2-sphere and do calculus on it as you would $\mathbb R^2$. However, you can split it into a couple flat patches that you know how to glue together and then differentiate and integrate as you please.
\end{example}

\begin{example}
  Configuration spaces of mechanical systems are manifolds. The configuration space of the planar pendulum is simply $S^1$ and $S^1\times S^1$ (a torus) for the double pendulum. The parameters of the configuration space are the degrees of freedom and the dimension of this manifold is the number of degrees of freedom.
\end{example}

\begin{example}
  To a point $x$ on a manifold, $M$, one can associate a vector tangent to that point living in the tangent space of the point, $T_xM$. The collection of all tangent spaces of all points on the manifold is again a manifold, called the tangent bundle, $TM$. A point on the tangent bundle is the data of a vector and the location on $M$ it sits at. Suppose $x_i$ is a local coordinate system on $M$ and $v_i$ components of a tangent vector in the same coordinate system. Then, $(x_i, v_i)$ are coordinates on $TM$. Since the tangent vector is associated to a derivative, one might right this more suggestively as $(x_i, dx_i)$ 
\end{example}

\begin{definition}
  Let $M$ be a manifold and $T_xM$ the tangent space over $x\in M$. The collection of inner products (like a dot product) on these vector spaces for all $x$ is the \emph{metric} on $M$. Like $M$ itself, the metric, $g$, is an object (a quadratic form) that is locally modeled on the euclidean dot product and glued together in a a particular way. 
\end{definition}
\section{The line element $ds$}

The metric allows us to talk about distances on arbitrary manifolds. For $(x^i, v^i)$ coordinates on $TM$, the metric is, in these local coordinates,
$$
ds^2 = g_{ij}(x)v^iv^j
$$
where $g_{ij}$ are the components of the \emph{metric tensor}. 
\begin{example}
  The length of a curve is 
  $$
\int ds = \int \sqrt{g_{ij}v^iv^j}
  $$
  In $\mathbb R^2$ $g_{ij}=\delta_{ij}$ so we would write $ds=\sqrt{dx^2+dy^2}$. We want to write this in terms of a Jacobian and pull out a differential element:
  $$
\int ds=\int\sqrt{dx^2+dy^2} = \int dx\sqrt{1+\frac{dy^2}{dx^2}}.
  $$
  Hopefully you recognize this as the length of a curve embedded in $\mathbb R^2$. In general, we have the object
  $$
  \int \sqrt{g_{ij}dx^i dx^j}
  $$
  computing the length of a curve parametrized by $x^i(\tau)$. By the chain rule, 

  $$
\int \sqrt{g_{ij}dx(\tau)^i dx(\tau)^j}=\int d\tau  \sqrt{g_{ij}\frac{dx^i}{d\tau}\frac{dx^j}{d\tau}}.
  $$
  Now, this looks like an action with Lagrangian
  $$
  \bar L=\sqrt{g_{ij}\frac{dx^i}{d\tau}\frac{dx^j}{d\tau}}.
  $$

  Since we are told to minimize the action functional, we can take 
  $$
L=g_{ij}\frac{dx^i}{d\tau}\frac{dx^j}{d\tau} = g_{ij}\dot x^i\dot x^j
  $$
  without changing the optimization problem - the square root will cause us problems otherwise. The E-L equations are (remember, the metric depends on $x$ in general)
  $$
  \frac{\partial g_{ij}}{\partial x^k}\dot x^i\dot x^j = \frac{d}{d\tau}\left\{2 g_{kj}\dot x^{j}\right\}=2g_{jk} \ddot x^j+2\frac{\partial g_{jk}}{\partial x^i}\dot x ^k \dot x^i
  $$
  It is useful to note 
  $$
2\frac{\partial g_{jk}}{\partial x^i}\dot x ^k \dot x^i = \frac{\partial g_{jk}}{\partial x^i}+\frac{\partial g_{ji}}{\partial x^k}.
  $$
  We will not see the utility of this here but it will keep us consistent with the formulation you will encounter in GR.

  The E-L equation becomes 
  $$
0=\frac{1}{2}\left(\frac{\partial g_{jk}}{\partial x^i}+\frac{\partial g_{ji}}{\partial x^k}\right)\dot x^i\dot x^j-\frac{\partial g_{ij}}{\partial x^k}\dot x^i\dot x^j+g_{jk} \ddot x^j.
  $$
  Defining (this is the Christoffel symbol)
  $$
\Gamma_{ijk}:=\left(\frac{\partial g_{jk}}{\partial x^i}+\frac{\partial g_{ji}}{\partial x^k}-\frac{\partial g_{ij}}{\partial x^k}\right)
  $$
  one obtains the geodesic equation
  \begin{empheq}[box=\tcbhighmath]{align*}
   g_{ij}\ddot x+\Gamma_{ijk}\dot x^i\dot x^j=0
\end{empheq}
The solution to this equation gives the curve of least distance on an arbitrary manifold equipped with metric $g$. It is also the path a light ray takes on a curved spacetime. 


\end{example}




\end{document}

\begin{empheq}[box=\tcbhighmath]{align*}
    F_r&=m(\ddot r-r\dot \theta^2-r\dot\phi^2\sin^2\theta)\\
  F_\theta&=m(2\dot r\dot \theta+r\ddot \theta-r\dot\phi^2\sin\theta\cos\theta)\\
  F_\phi&=m(r\ddot \phi \sin \theta +2\dot r \dot \phi \sin \theta +2r\dot\theta \dot \phi\cos\theta)
\end{empheq}

   