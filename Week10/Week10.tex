\documentclass{article}
\usepackage{amsmath}
\usepackage{amssymb}
\usepackage{graphicx}
\usepackage{amsthm}
\usepackage{empheq}
\usepackage{float}
\usepackage[utf8]{inputenc}
\usepackage{amssymb}
\usepackage[most]{tcolorbox}
\usepackage{hyperref}
\newtheorem*{theorem}{Thm}
\newtheorem*{proposition}{Prop}
\theoremstyle{definition}
\newtheorem*{definition}{Defn}

\theoremstyle{remark}
\newtheorem*{remark}{Remark}

\theoremstyle{remark}
\newtheorem*{example}{Example}


\newtcbox{\mymath}[1][]{%
    nobeforeafter, math upper, tcbox raise base,
    enhanced, colframe=blue!30!black,
    colback=blue!30, boxrule=1pt,
    #1}
\title{Non-intertial motion - Rotating frames}




\begin{document}

\maketitle

We start with noting a direct corollary of one of the axioms of classical mechanics:
\begin{theorem}
  Given a motion $\vec x: \mathbb R\to \mathbb R^3$, one can always find a frame in which $\vec F = m\ddot{\vec x}$. Such frames are inertial frames.
\end{theorem}
That is, a motion specifies a system of \emph{inertial} forces and vice versa. Note, however, that it is not always easy to work with inertial frames. As an example, try to solve the first discussion problem in inertial coordinates. 


\begin{theorem}
Let $\vec x_{NI}$ be a non-inertial motion. Then, in this non-inertial frame there is an effective system of forces such that
$$
\vec F_{eff} = m\ddot{\vec x}_{NI}.
$$ 
where $\vec F_{eff} = \vec F_{real}+\vec F_{fake}$. 
\end{theorem}

\begin{proof}
  $\vec x_{NI}$ is a trajectory and has the usual nice properties one expects of a solution to a second order ODE. So, (by Picard's theorem), such an  $\vec x_{NI}$ solves \emph{some} second order ODE which we may massage in to the form of Newton's second law. Clearly, the system of forces is not just the physical/real forces as this would contradict the previous theorem. Hence, the given decomposition.
\end{proof}

Let us specialize to rotating frames and rework this idea more explicitly.

\begin{proposition}
  Let $x^{(,)}$ denote motion in the (non-)inertial frame which we assume to be rotating. They are related through relation $x'=x+x_0$ ($x_0$ denoting origin of rotating frame). We have infinitesimally
  $$
  \dot x' = \dot x +\dot x_0+\omega \times x 
  $$
  This is true in general for any vector $\vec x$, not just positions/motion.
\end{proposition}
\begin{proof}
  I'm sure you'll see this result derived in class.
\end{proof}
For reference, here is this result expressed more cleanly:
\begin{empheq}[box=\tcbhighmath]{align*}
    \vec v_{I} = \vec v_0+\vec v_{NI}+\vec \omega \times \vec x
\end{empheq}
$\vec \omega$ is the rotation frequency of the frame.

\begin{theorem}
  Let $\vec x_{NI}$ be a motion in a rotating frame. Then, in this non-inertial frame there is an effective system of forces such that
$$
\vec F_{eff} = m\ddot{\vec x}_{NI}.
$$ 
where 
\begin{empheq}[box=\tcbhighmath]{align*}
\vec F_{eff} = \vec F_{real}-m\dot v_0-m\dot \omega \times x_{NI}-m\omega \times(\omega \times x_{NI})-2m\omega \times v_{NI}\end{empheq}. 
\end{theorem}

\begin{proof}
  Start with $F=m\dot v_{I}$ and carefully differentiate in inertial frame to obtain $F_{fake}$.
\end{proof}

\begin{remark}
  Note that we can write $\dot v_0 = \omega \times v_0=\omega \times (\omega\times x_0)$ by our proposition applied twice. 
\end{remark}

\begin{definition}
  The force of the form $2m\omega \times v_{NI}$ felt by a rotating observer is the \emph{Coriolis force}. $-m\omega \times(\omega \times x_{NI})$ is the \emph{centrifugal force}. 
\end{definition}

\begin{tcolorbox}[title=Motion relative to rotating Earth]
  First, note that the rotation of the Earth about its center is almost constant in time, $\dot \omega =0$. Also note in general that $\vec F_{real}$ is very complicated. We often write $\vec F_{real} = \vec f+m g_{I}$ and we don't worry about the $\vec f$. Then,

  $$
  F_{eff} = f +mg_{I}-m\dot v_0-m\omega \times(\omega \times x_{NI})-2m\omega \times v_{NI}.
$$
We can replace the acceleration of the origin $\dot v_0$ with $\omega \times (\omega\times x_0)$ and get 
$$
F_{eff} = f +mg_{I}-m\dot v_0-m\omega \times(\omega \times( x_{NI}+x_0))-2m\omega \times v_{NI}.
$$ 
The action of the centrifugal force is usually absorbed into an effective non-inertial gravitational acceleration
$$
g_{NI}= g_I - \omega \times(\omega\times(x_{NI}+x_0)).
$$
Finally,
\begin{empheq}[box=\tcbhighmath]{align*}
F_{eff}=f+mg_{NI}-2m\omega\times v_{NI}
\end{empheq}

In practice, you will usually only care about the Coriolis part of this expression.
\end{tcolorbox}
\end{document}

\begin{empheq}[box=\tcbhighmath]{align*}
    F_r&=m(\ddot r-r\dot \theta^2-r\dot\phi^2\sin^2\theta)\\
  F_\theta&=m(2\dot r\dot \theta+r\ddot \theta-r\dot\phi^2\sin\theta\cos\theta)\\
  F_\phi&=m(r\ddot \phi \sin \theta +2\dot r \dot \phi \sin \theta +2r\dot\theta \dot \phi\cos\theta)
\end{empheq}

   